%%%%%%%%%%%%%%%%%%%%%%%%%%%%%%%%%%%%%%%%%
% "ModernCV" CV and Cover Letter
% LaTeX Template
% Version 1.11 (19/6/14)
%
% This template has been downloaded from:
% http://www.LaTeXTemplates.com
%
% Original author:
% Xavier Danaux (xdanaux@gmail.com)
%
% License:
% CC BY-NC-SA 3.0 (http://creativecommons.org/licenses/by-nc-sa/3.0/)
%
% Important note:
% This template requires the moderncv.cls and .sty files to be in the same 
% directory as this .tex file. These files provide the resume style and themes 
% used for structuring the document.
%
%%%%%%%%%%%%%%%%%%%%%%%%%%%%%%%%%%%%%%%%%

%----------------------------------------------------------------------------------------
%	PACKAGES AND OTHER DOCUMENT CONFIGURATIONS
%----------------------------------------------------------------------------------------


\documentclass[11pt,a4paper,sans]{moderncv} % Font sizes: 10, 11, or 12; paper sizes: a4paper, letterpaper, a5paper, legalpaper, executivepaper or landscape; font families: sans or roman


\moderncvstyle{classic} % CV theme - options include: 'casual' (default), 'classic', 'oldstyle' and 'banking'
\moderncvcolor{purple} % CV color - options include: 'blue' (default), 'orange', 'green', 'red', 'purple', 'grey' and 'black'

\usepackage{lipsum} % Used for inserting dummy 'Lorem ipsum' text into the template

\usepackage[scale=0.9]{geometry} % Reduce document margins
%\setlength{\hintscolumnwidth}{3cm} % Uncomment to change the width of the dates column
\setlength{\makecvtitlenamewidth}{10cm} % For the 'classic' style, uncomment to adjust the width of the space allocated to your name


%----------------------------------------------------------------------------------------
%	NAME AND CONTACT INFORMATION SECTION
%----------------------------------------------------------------------------------------

\firstname{Sudhir Shinde} % Your first name
%\familyname{Bhardwaj} % Your last name

% All information in this block is optional, comment out any lines you don't need
\title{Curriculum Vitae}
%\address{Address}{City, State Zip}
\mobile{(+91) 9404521215}
\email{sudhirshinde58@gmail.com}
% \homepage{  https://www.linkedin.com/in/sudhir-shinde-198088118/} {LinkedIn} % The first argument is the url for the clickable link, the second argument is the url displayed in the template - this allows special characters to be displayed such as the tilde in this example
% \homepage{https://sites.google.com/view/sikka-anmol/home}{Homepage}
%\extrainfo{DOB: September 15, 1997}
%\photo[70pt][0.4pt]{pictures/House} % The first bracket is the picture height, the second is the thickness of the frame around the picture (0pt for no frame)
%\quote{"A witty and playful quotation" - John Smith}
%----------------------------------------------------------------------------------------

\begin{document}

\makecvtitle % Print the CV title
%----------------------------------------------------------------------------------------
%	EDUCATION SECTION
%----------------------------------------------------------------------------------------

\section{Education}

\cventry{2017--2020}{M.Tech}{}{}{\textit{Computer Aided Design and Manufacturing, \link[Indian Institute of Technology (IIT), Bombay (India)]{http://www.iitb.ac.in/}, CPI \textbf{9.45/10}}}{}

\cventry{2016}{B.Tech}{}{}{\textit{\link[Savitribai Phule Pune University]{http://www.unipune.ac.in/}, Pune, India \textbf{79.67\%}}}{}

% \cventry{2013}{Diploma}{}{}{\textit{\link[PVP Institute of Technology]{http://pvpitsangli.edu.in/web/}, Sangli, India 82.99\%}}{}


%----------------------------------------------------------------------------------------
%	INTERESTS SECTION
%----------------------------------------------------------------------------------------


% \renewcommand{\listitemsymbol}{-~} % Changes the symbol used for lists
% \cvlistdoubleitem{Piano}{}
% \cvlistitem{Baseball}

%----------------------------------------------------------------------------------------
%	Achievements SECTION
%----------------------------------------------------------------------------------------

\section{Research summary}

\cvitem{M.Tech Research}{\textbf{Defeaturing of 3D CAD Models using Deep Learning}
\begin{itemize}
    \item Worked with \link[\textbf{Prof S.S. Pande}]{https://www.me.iitb.ac.in/~sspande/} to create Deep Learning model for feature recognition and to decide the present/absent of feature for improvement in design and saving computational time
    \item Developed a system to generate 10K 3D models with distinct topological features in \textbf{Python}
    \item Feature recognition and extraction is done by unsupervised ML algorithm (DBSCAN)
    % \item Extracting features from CAD models using concept based \textbf{3D CNN}
    % \item  reduced simulation computational time using \textbf{autoencoder} and \textbf{PCA}
    \item Developed a GUI in \textbf{Visual Studio} using MFC application in \textbf{C++} for feature recognition
\end{itemize}
}
% Currently working on defeaturing of CAD Models using 3D convolution neural network. Developing system to generate 10K 3D models with distinct topological features in Python. Feature recognition and extraction is done by unsupervised ML algorithm (DBSCAN). Aiming to reduce simulation computational time using autoencoder and PCA}

% \cvitem{M.Tech Seminar} {Presented seminar on application of Machine Learning in CAD/CAM domain}
% \cvitem{2017-18}{Headed the Enactus IGDTU as the\textbf{Project Leader}of Dharini project impacting the lives of \textbfup{1,500 underprivileged women}directly and indirectly}

% \cvitem{2017-18}{Fecilitated as the top 20 teams in \textbf{Enactus National Competition} for exceptional entrepreneurial action and social impact in field of \textbf{Menstrual Hygiene} and upliftment of \textbf{Rohingaya community} organised by KPMG India}
% \cvitem{2016-2017}{Recipient of the \textbf{Director's Special Award } and headed the student council as 'Prefect' of the school}
% \cvitem{2013}{ Felicitated with the prestigious \textbf{Aqua Award} for movie making on water conservation organised by the PVR CineArt and the Aqua Foundation}
% \cvitem{2014}{\textbf{State Winner of French Spell Bee}, Advanced Level organised by Paris based Le Frehindi Organisation}


% FABGUARD SERVICES

\section{Internship Experience}

\cvitem{May'19-Aug'19}{\textbf{Machine Learning Engineer Intern, Fabguard Services}
\begin{itemize}
    \item Designed a concept based deep learning framework to identify anomalies in objects, trajectories, and actions \& trained in \textbf{PyTorch} framework.
    % \item Increased \textbf{productivity by 5\%} through implementing project, \textbf{optimization} of tire components
    % \item Extracting features from CAD models using concept based \textbf{3D CNN}
    % \item Hands on experience to do the simulation and post-processing of tire with scripting in \textbf{Abaqus}
    \item Studied deep learning based latest research papers to improve the accuracy of the models
\end{itemize}
}



\cvitem{Jan'19-Apr'19}{\textbf{Research Consultant, WorldQuant Research India Pvt. Ltd.}
\begin{itemize}
    \item Developed mathematical models (alphas) based on long-short trading strategies for weighing stocks in a portfolio in order to optimize the same by attaining high returns with low volatility
    \item Researched various algorithms and strategies for enhancing the efficiency of alphas
\end{itemize}
}





% Currently working on defeaturing of CAD Models using 3D convolution neural network. Developing system to generate 10K 3D models with distinct topological features in Python. Feature recognition and extraction is done by unsupervised ML algorithm (DBSCAN). Aiming to reduce simulation computational time using autoencoder and PCA}

% \cvitem{M.Tech Seminar} {Presented seminar on application of Machine Learning in CAD/CAM domain}



%----------------------------------------------------------------------------------------
%	AWARDS SECTION
%----------------------------------------------------------------------------------------
%\section{Certifications}

%\cventry{April 2015 -- April 2017}{Certified LabVIEW Associate Developer}{\textsc{National Instruments}}{}{}{}


%----------------------------------------------------------------------------------------
%	Research SECTION
%----------------------------------------------------------------------------------------


%----------------------------------------------------------------------------------------
%	Techincal Projects SECTION
%----------------------------------------------------------------------------------------





%----------------------------------------------------------------------------------------
%	Course Projects SECTION
%----------------------------------------------------------------------------------------

\section{Course Projects}
\cvitem{Machine Learning}{\textbf{Machine Learning based Image Classification System to Analyze Changing Fashion Trends}
\begin{itemize}
    \item Developed the shirt classification system using \textbf{KNN, SVM, CNN} with \textbf{Scikit-Learn} and \textbf{TensorFlow}
    \item Pre-processing involved face detection \& neck region feature to create dataset using \textbf{OpenCV}
    \item Achieved the accuracy of 84\% for all the classes using \textbf{AlexNet} architecture as a base framework
\end{itemize}
}
% \cvitem{}{\textit{Prof. Ganesh Ramakrishnan, Department of Computer Science}}

\cvitem{Microsoft Codefundo++}{\textbf{Development of ML Algorithm for Flood Prediction on Azure Cloud Service}
\begin{itemize}
    \item Dataset gleaned from Indian meteorology websites comprised of historical rainfall \& altitude
    % \item Successfully completed all three stages and implemented web application on Azure Cloud Services
    \item Designed and deployed ML workflow (flood prediction) on Azure Cloud Services
\end{itemize}
}
% \cvitem{}{\textit{Microsoft Codefundo++}}

\cvitem{Parallel Programming}{\textbf{GPU accelerated implementation of Machine Learning algorithms using CUDA}
\begin{itemize}
    \item Implemented parallelization of k-fold cross validation for regression and classification in CUDA \textbf{C++}
    \item Defined \textbf{CUDA} kernels for Linear Regression and Logistic Regression using Gradient Descent algorithm
    \item Achieved speed up of \textbf{3.5X} for regression and \textbf{2.5X} for classification as compared with serial code
\end{itemize}
}
% \cvitem{}{\textit{High Performance Scientific Computing, Advisor: Prof. Shiva Gopalakrishnan}}

% \cvitem{Machine Learning}{\textbf{Implementation of Neural Network based Classifier}
% \begin{itemize}
%     \item Implemented Neural Network from scratch to classify the Facebook comments into 5 categories
%     \item Developed NN architecture using \textbf{NumPy, Pandas} library and trained using Back Propagation Algorithm
%     \item Improved accuracy by using different activation functions along with hyper-parameters tuning
% \end{itemize}
% }

\cvitem{Web Development}{\textbf{Personal Blog Web App in Django.}
\begin{itemize}
    % \item Created a personal blog web application in \textbf{Django (Python)}
    \item The application help to create/edit/delete blog using QuerySets and display on the website
    \item Implemented user-authentication features (login, logout, register) from scratch
\end{itemize}
}

% \cvitem{}{\textit{Foundations of Machine Learning, Advisor: Prof. Ganesh Ramakrishnan}}

\cvitem{Web \& Coding club, IIT Bombay}{\textbf{MeshNet: Mesh Neural Network for 3D Shape Representation}
\begin{itemize}
    \item Implemented a research paper for 3D shape classification which takes data in mesh format and improved its accuracy to \textbf{89.6\%}
    % \item Compared there performance using R-square, Adj. R, P-value, F-Stat and root mean square error
\end{itemize}
}
% \cvitem{}{\textit{Engineering Data Mining and Applications, Advisor: Prof. Vinay Kulkarni}}
%----------------------------------------------------------------------------------------
%	Institute Positions SECTION
%----------------------------------------------------------------------------------------



\section{Technical Projects}
\cvitem{Innovation Cell, IIT Bombay}{\textbf{Mahindra Rise Driverless Car Challenge}
\begin{itemize}
    \item Part of a team of 20 members aiming to build Self Driving Car; India's 1st driverless car
    \item One of the 11 finalists out of 259 teams (IV Level) and received a\textbf{ Mahindra E2O Car} for further development
    \item Headed the mechatronics subsystem to design mechanisms to mount \textbf{3D LIDAR}, Camera on the car
\end{itemize}
}

% \cvitem{}{\textit{Innovation Cell, IIT Bombay}}

\cvitem{Computer Graphics}{\textbf{Voxelization of 3D model and Cutting Forces Prediction}
\begin{itemize}
    \item To develop voxelization algorithm of 3D CAD model for visualization using \textbf{OpenGL}
    \item Predicted the cutting forces and material removal rate during machining using voxelized CAD model
\end{itemize}
}
% \cvitem{}{\textit{Computer Graphics and Product Modelling, Advisor: Prof. S.S.Pande}}

% \cvitem{August 2017}{\textbf{Internet Banking App:Prototype}}
% \cvitem{}{\textit{Developed for Esya,IIITD}}







\section{Positions of Responsibility}

\cvitem{July'18-May'20}{\textbf{Teaching Assistant, IIT Bombay}
\begin{itemize}
    \item served 3 times as a TA for Computer Graphics and Product Modelling and Computer Numerical Control and Programming course at IIT Bombay
    \item Assisted Bachelor and Master students to clear their difficulties, also helping the professor in evaluation
\end{itemize}
}
% \cvitem{}{\textit{Computer Graphics and Product Modelling}}

\cvitem{Jan'19-May'20}{\textbf{Mentor ITSP, IIT Bombay}
\begin{itemize}
    \item Guided \textbf{20 students} on the topics OCR, handwritten character recognition using Deep Learning
    \item Provided the basic training of Python and Machine Learning to students
\end{itemize}
}
% \cvitem{}{\textit{Guided 8 students on the topics OCR recognition, handwritten character recognition using Deep Learning}}
\cvitem{May'19-Apr'20}{\textbf{Campus Ambassador, InterviewBit}
\begin{itemize}
    \item Organized coding competitions to help students for campus placement preparation
\end{itemize}
}
% \cvitem{}{\textit{Organized coding competitions to help students for campus placement preparation}}


%----------------------------------------------------------------------------------------
%	Skills SECTION
%----------------------------------------------------------------------------------------

\section{Technical Skills}
%
\cvitem{Programming}{C, C++, Python}
\cvitem{Tools}{PyTorch, TensorFlow, MATLAB, Scikit-learn, Django, OpenCV, Git, \LaTeX}
\cvitem{Web Development}{FrontEnd- HTML, CSS}
% \cvitem{Others}{Machine Learning, Data Science, Deep Learning}

%----------------------------------------------------------------------------------------
%	Coursework SECTION
%----------------------------------------------------------------------------------------

\section{Major Courses}

%\renewcommand{\listitemsymbol}{-~} % Changes the symbol used for lists

\cvitem{Core Courses}{Data Structures \& Algorithms, Foundations of Machine Learning, Engineering Data Mining and Applications, High Performance Scientific Computing, Computer Graphics \& Product Modelling, Robotics}


% \cvitem{Lab Courses}{Database Management Systems Lab, Data Structure Lab, Object Oriented Programming using C++ and JAVA Lab, Analog and Digital Electronics Lab, Computer Organization and Architecture Lab*, Analysis and Design of Algorithms Lab Operating System Lab (using LINUX as Case Study)*,Object Oriented Software Engineering Lab*}

% \cvitem{}{}
% \cvitem{}{* -To be completed by Apr 2019}
%\cvlistdoubleitem{Basics of Electricity & Magnetism}{Computer Programming and Utilization}

%----------------------------------------------------------------------------------------
%	COMMUNICATION SKILLS SECTION
%----------------------------------------------------------------------------------------


%----------------------------------------------------------------------------------------
%	LANGUAGES SECTION
%----------------------------------------------------------------------------------------

\section{Achievements \& Extracurricular Activities}

% \cvitemwithcomment{2018}{Achieved \textbf{Rank 1 on Kaggle} among 112 students for Machine Learning Challenge}{}
\cvitemwithcomment{2016}{Secured \textbf{Department Rank 1} among 210 students of UG 2016 batch}{}
\cvitemwithcomment{2018}{Scored \textbf{99.31} percentile in GATE 2018 among 194,496 candidates}{}
\cvitemwithcomment{2019}{Secured \textbf{Gold level} position in the \textbf{WorldQuant Challenge} organised by WorldQuant VRC}{}
\cvitemwithcomment{2013}{Secured \textbf{Department Rank 3} in Diploma of 2012 batch}{}
\cvitemwithcomment{2018}{Represented IIT Bombay in \textbf{Microsoft Codefundo++}}{}
\cvitemwithcomment{2019}{Attended 3 days \textbf{GPU} bootcamp using CUDA conducted by NVIDIA}{}
% \cvitemwithcomment{2019}{Completed \textbf{Neural Networks and Deep Learning} course by DeepLearning.ai on Coursera}{}
\cvitemwithcomment{2019}{Completed \textbf{Databases with python} course on Coursera}{}
\cvitemwithcomment{2018}{Volunteered for \textbf{Python Workshop} conducted by PG Academic Council}{}
\cvitemwithcomment{2015}{Participated in AVISHKAR Zonal level Research Project Competition organized by Pune University}{}


% \section{Links}

% \cvitem{Profiles}{\link[GitHub]{https://github.com/sudhir5595}, \link[LeetCode]{https://leetcode.com/sudhirshinde58/}}

% \cvitem{LeetCode}{}

%\cvitemwithcomment{Dutch}{Basic}{Basic words and phrases only}


%----------------------------------------------------------------------------------------
%	COVER LETTER
%----------------------------------------------------------------------------------------

% To remove the cover letter, comment out this entire block

%\clearpage

%\recipient{HR Department}{Corporation\\123 Pleasant Lane\\12345 City, State} % Letter recipient
%\date{\today} % Letter date
%\opening{Dear Sir or Madam,} % Opening greeting
%\closing{Sincerely yours,} % Closing phrase
%\enclosure[Attached]{curriculum vit\ae{}} % List of enclosed documents

%\makelettertitle % Print letter title

%\lipsum[1-3] % Dummy text

%\makeletterclosing % Print letter signature

%----------------------------------------------------------------------------------------

\end{document}



